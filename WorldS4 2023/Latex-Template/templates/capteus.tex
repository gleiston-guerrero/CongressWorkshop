%%%%%%%%%%%%%%%%%%%% author.tex %%%%%%%%%%%%%%%%%%%%%%%%%%%%%%%%%%%
%
% sample root file for your "contribution" to a proceedings volume
%
% Use this file as a template for your own input.
%
%%%%%%%%%%%%%%%% Springer %%%%%%%%%%%%%%%%%%%%%%%%%%%%%%%%%%


\documentclass{svproc}
%
% RECOMMENDED %%%%%%%%%%%%%%%%%%%%%%%%%%%%%%%%%%%%%%%%%%%%%%%%%%%
%

% to typeset URLs, URIs, and DOIs
\usepackage{url}
\def\UrlFont{\rmfamily}

\begin{document}
\mainmatter              % start of a contribution
%
\title{Hamiltonian Mechanics unter besonderer Ber\"ucksichtigung der
h\"ohereren Lehranstalten}
%
\titlerunning{Hamiltonian Mechanics}  % abbreviated title (for running head)
%                                     also used for the TOC unless
%                                     \toctitle is used
%
\author{Ivar Ekeland\inst{1} \and Roger Temam\inst{2}
Jeffrey Dean \and David Grove \and Craig Chambers \and Kim~B.~Bruce \and
Elsa Bertino}
%
\authorrunning{Ivar Ekeland et al.} % abbreviated author list (for running head)
%
%%%% list of authors for the TOC (use if author list has to be modified)
\tocauthor{Ivar Ekeland, Roger Temam, Jeffrey Dean, David Grove,
Craig Chambers, Kim B. Bruce, and Elisa Bertino}
%
\institute{Princeton University, Princeton NJ 08544, USA,\\
\email{I.Ekeland@princeton.edu},\\ WWW home page:
\texttt{http://users/\homedir iekeland/web/welcome.html}
\and
Universit\'{e} de Paris-Sud,
Laboratoire d'Analyse Num\'{e}rique, B\^{a}timent 425,\\
F-91405 Orsay Cedex, France}

\maketitle              % typeset the title of the contribution

\begin{abstract}
The abstract should summarize the contents of the paper
using at least 70 and at most 150 words. It will be set in 9-point
font size and be inset 1.0 cm from the right and left margins.
There will be two blank lines before and after the Abstract. \dots
% We would like to encourage you to list your keywords within
% the abstract section using the \keywords{...} command.
\keywords{computational geometry, graph theory, Hamilton cycles}
\end{abstract}
%
\section{Fixed-Period Problems: The Sublinear Case}
%
With this chapter, the preliminaries are over, and we begin the search
for periodic solutions to Hamiltonian systems. All this will be done in
the convex case; that is, we shall study the boundary-value problem
\begin{eqnarray*}
  \dot{x}&=&JH' (t,x)\\
  x(0) &=& x(T)
\end{eqnarray*}
with $H(t,\cdot)$ a convex function of $x$, going to $+\infty$ when
$\left\|x\right\| \to \infty$.

%
\subsection{Autonomous Systems}
%
In this section, we will consider the case when the Hamiltonian $H(x)$
is autonomous. For the sake of simplicity, we shall also assume that it
is $C^{1}$.

We shall first consider the question of nontriviality, within the
general framework of
$\left(A_{\infty},B_{\infty}\right)$-subquadratic Hamiltonians. In
the second subsection, we shall look into the special case when $H$ is
$\left(0,b_{\infty}\right)$-subquadratic,
and we shall try to derive additional information.
%
\subsubsection{The General Case: Nontriviality.}
%
We assume that $H$ is
$\left(A_{\infty},B_{\infty}\right)$-sub\-qua\-dra\-tic at infinity,
for some constant symmetric matrices $A_{\infty}$ and $B_{\infty}$,
with $B_{\infty}-A_{\infty}$ positive definite. Set:
\begin{eqnarray}
\gamma :&=&{\rm smallest\ eigenvalue\ of}\ \ B_{\infty} - A_{\infty} \\
  \lambda : &=& {\rm largest\ negative\ eigenvalue\ of}\ \
  J \frac{d}{dt} +A_{\infty}\ .
\end{eqnarray}

Theorem~\ref{ghou:pre} tells us that if $\lambda +\gamma < 0$, the
boundary-value problem:
\begin{equation}
\begin{array}{rcl}
  \dot{x}&=&JH' (x)\\
  x(0)&=&x (T)
\end{array}
\end{equation}
has at least one solution
$\overline{x}$, which is found by minimizing the dual
action functional:
\begin{equation}
  \psi (u) = \int_{o}^{T} \left[\frac{1}{2}
  \left(\Lambda_{o}^{-1} u,u\right) + N^{\ast} (-u)\right] dt
\end{equation}
on the range of $\Lambda$, which is a subspace $R (\Lambda)_{L}^{2}$
with finite codimension. Here
\begin{equation}
  N(x) := H(x) - \frac{1}{2} \left(A_{\infty} x,x\right)
\end{equation}
is a convex function, and
\begin{equation}
  N(x) \le \frac{1}{2}
  \left(\left(B_{\infty} - A_{\infty}\right) x,x\right)
  + c\ \ \ \forall x\ .
\end{equation}

%
\begin{proposition}
Assume $H'(0)=0$ and $ H(0)=0$. Set:
\begin{equation}
  \delta := \liminf_{x\to 0} 2 N (x) \left\|x\right\|^{-2}\ .
  \label{eq:one}
\end{equation}

If $\gamma < - \lambda < \delta$,
the solution $\overline{u}$ is non-zero:
\begin{equation}
  \overline{x} (t) \ne 0\ \ \ \forall t\ .
\end{equation}
\end{proposition}
%
\begin{proof}
Condition (\ref{eq:one}) means that, for every
$\delta ' > \delta$, there is some $\varepsilon > 0$ such that
\begin{equation}
  \left\|x\right\| \le \varepsilon \Rightarrow N (x) \le
  \frac{\delta '}{2} \left\|x\right\|^{2}\ .
\end{equation}

It is an exercise in convex analysis, into which we shall not go, to
show that this implies that there is an $\eta > 0$ such that
\begin{equation}
  f\left\|x\right\| \le \eta
  \Rightarrow N^{\ast} (y) \le \frac{1}{2\delta '}
  \left\|y\right\|^{2}\ .
  \label{eq:two}
\end{equation}

\begin{figure}
\vspace{2.5cm}
\caption{This is the caption of the figure displaying a white eagle and
a white horse on a snow field}
\end{figure}

Since $u_{1}$ is a smooth function, we will have
$\left\|hu_{1}\right\|_\infty \le \eta$
for $h$ small enough, and inequality (\ref{eq:two}) will hold,
yielding thereby:
\begin{equation}
  \psi (hu_{1}) \le \frac{h^{2}}{2}
  \frac{1}{\lambda} \left\|u_{1} \right\|_{2}^{2} + \frac{h^{2}}{2}
  \frac{1}{\delta '} \left\|u_{1}\right\|^{2}\ .
\end{equation}

If we choose $\delta '$ close enough to $\delta$, the quantity
$\left(\frac{1}{\lambda} + \frac{1}{\delta '}\right)$
will be negative, and we end up with
\begin{equation}
  \psi (hu_{1}) < 0\ \ \ \ \ {\rm for}\ \ h\ne 0\ \ {\rm small}\ .
\end{equation}

On the other hand, we check directly that $\psi (0) = 0$. This shows
that 0 cannot be a minimizer of $\psi$, not even a local one.
So $\overline{u} \ne 0$ and
$\overline{u} \ne \Lambda_{o}^{-1} (0) = 0$. \qed
\end{proof}
%
\begin{corollary}
Assume $H$ is $C^{2}$ and
$\left(a_{\infty},b_{\infty}\right)$-subquadratic at infinity. Let
$\xi_{1},\allowbreak\dots,\allowbreak\xi_{N}$  be the
equilibria, that is, the solutions of $H' (\xi ) = 0$.
Denote by $\omega_{k}$
the smallest eigenvalue of $H'' \left(\xi_{k}\right)$, and set:
\begin{equation}
  \omega : = {\rm Min\,} \left\{\omega_{1},\dots,\omega_{k}\right\}\ .
\end{equation}
If:
\begin{equation}
  \frac{T}{2\pi} b_{\infty} <
  - E \left[- \frac{T}{2\pi}a_{\infty}\right] <
  \frac{T}{2\pi}\omega
  \label{eq:three}
\end{equation}
then minimization of $\psi$ yields a non-constant $T$-periodic solution
$\overline{x}$.
\end{corollary}
%

We recall once more that by the integer part $E [\alpha ]$ of
$\alpha \in \bbbr$, we mean the $a\in \bbbz$
such that $a< \alpha \le a+1$. For instance,
if we take $a_{\infty} = 0$, Corollary 2 tells
us that $\overline{x}$ exists and is
non-constant provided that:

\begin{equation}
  \frac{T}{2\pi} b_{\infty} < 1 < \frac{T}{2\pi}
\end{equation}
or
\begin{equation}
  T\in \left(\frac{2\pi}{\omega},\frac{2\pi}{b_{\infty}}\right)\ .
  \label{eq:four}
\end{equation}

%
\begin{proof}
The spectrum of $\Lambda$ is $\frac{2\pi}{T} \bbbz +a_{\infty}$. The
largest negative eigenvalue $\lambda$ is given by
$\frac{2\pi}{T}k_{o} +a_{\infty}$,
where
\begin{equation}
  \frac{2\pi}{T}k_{o} + a_{\infty} < 0
  \le \frac{2\pi}{T} (k_{o} +1) + a_{\infty}\ .
\end{equation}
Hence:
\begin{equation}
  k_{o} = E \left[- \frac{T}{2\pi} a_{\infty}\right] \ .
\end{equation}

The condition $\gamma < -\lambda < \delta$ now becomes:
\begin{equation}
  b_{\infty} - a_{\infty} <
  - \frac{2\pi}{T} k_{o} -a_{\infty} < \omega -a_{\infty}
\end{equation}
which is precisely condition (\ref{eq:three}).\qed
\end{proof}
%

\begin{lemma}
Assume that $H$ is $C^{2}$ on $\bbbr^{2n} \setminus \{ 0\}$ and
that $H'' (x)$ is non-de\-gen\-er\-ate for any $x\ne 0$. Then any local
minimizer $\widetilde{x}$ of $\psi$ has minimal period $T$.
\end{lemma}
%
\begin{proof}
We know that $\widetilde{x}$, or
$\widetilde{x} + \xi$ for some constant $\xi
\in \bbbr^{2n}$, is a $T$-periodic solution of the Hamiltonian system:
\begin{equation}
  \dot{x} = JH' (x)\ .
\end{equation}

There is no loss of generality in taking $\xi = 0$. So
$\psi (x) \ge \psi (\widetilde{x} )$
for all $\widetilde{x}$ in some neighbourhood of $x$ in
$W^{1,2} \left(\bbbr / T\bbbz ; \bbbr^{2n}\right)$.

But this index is precisely the index
$i_{T} (\widetilde{x} )$ of the $T$-periodic
solution $\widetilde{x}$ over the interval
$(0,T)$, as defined in Sect.~2.6. So
\begin{equation}
  i_{T} (\widetilde{x} ) = 0\ .
  \label{eq:five}
\end{equation}

Now if $\widetilde{x}$ has a lower period, $T/k$ say,
we would have, by Corollary 31:
\begin{equation}
  i_{T} (\widetilde{x} ) =
  i_{kT/k}(\widetilde{x} ) \ge
  ki_{T/k} (\widetilde{x} ) + k-1 \ge k-1 \ge 1\ .
\end{equation}

This would contradict (\ref{eq:five}), and thus cannot happen.\qed
\end{proof}
%
\paragraph{Notes and Comments.}
The results in this section are a
refined version of \cite{smit:wat};
the minimality result of Proposition
14 was the first of its kind.

To understand the nontriviality conditions, such as the one in formula
(\ref{eq:four}), one may think of a one-parameter family
$x_{T}$, $T\in \left(2\pi\omega^{-1}, 2\pi b_{\infty}^{-1}\right)$
of periodic solutions, $x_{T} (0) = x_{T} (T)$,
with $x_{T}$ going away to infinity when $T\to 2\pi \omega^{-1}$,
which is the period of the linearized system at 0.

\begin{table}
\caption{This is the example table taken out of {\it The
\TeX{}book,} p.\,246}
\begin{center}
\begin{tabular}{r@{\quad}rl}
\hline
\multicolumn{1}{l}{\rule{0pt}{12pt}
                   Year}&\multicolumn{2}{l}{World population}\\[2pt]
\hline\rule{0pt}{12pt}
8000 B.C.  &     5,000,000& \\
  50 A.D.  &   200,000,000& \\
1650 A.D.  &   500,000,000& \\
1945 A.D.  & 2,300,000,000& \\
1980 A.D.  & 4,400,000,000& \\[2pt]
\hline
\end{tabular}
\end{center}
\end{table}
%
\begin{theorem} [Ghoussoub-Preiss]\label{ghou:pre}
Assume $H(t,x)$ is
$(0,\varepsilon )$-subquadratic at
infinity for all $\varepsilon > 0$, and $T$-periodic in $t$
\begin{equation}
  H (t,\cdot )\ \ \ \ \ {\rm is\ convex}\ \ \forall t
\end{equation}
\begin{equation}
  H (\cdot ,x)\ \ \ \ \ {\rm is}\ \ T{\rm -periodic}\ \ \forall x
\end{equation}
\begin{equation}
  H (t,x)\ge n\left(\left\|x\right\|\right)\ \ \ \ \
  {\rm with}\ \ n (s)s^{-1}\to \infty\ \ {\rm as}\ \ s\to \infty
\end{equation}
\begin{equation}
  \forall \varepsilon > 0\ ,\ \ \ \exists c\ :\
  H(t,x) \le \frac{\varepsilon}{2}\left\|x\right\|^{2} + c\ .
\end{equation}

Assume also that $H$ is $C^{2}$, and $H'' (t,x)$ is positive definite
everywhere. Then there is a sequence $x_{k}$, $k\in \bbbn$, of
$kT$-periodic solutions of the system
\begin{equation}
  \dot{x} = JH' (t,x)
\end{equation}
such that, for every $k\in \bbbn$, there is some $p_{o}\in\bbbn$ with:
\begin{equation}
  p\ge p_{o}\Rightarrow x_{pk} \ne x_{k}\ .
\end{equation}
\qed
\end{theorem}
%
\begin{example} [{{\rm External forcing}}]
Consider the system:
\begin{equation}
  \dot{x} = JH' (x) + f(t)
\end{equation}
where the Hamiltonian $H$ is
$\left(0,b_{\infty}\right)$-subquadratic, and the
forcing term is a distribution on the circle:
\begin{equation}
  f = \frac{d}{dt} F + f_{o}\ \ \ \ \
  {\rm with}\ \ F\in L^{2} \left(\bbbr / T\bbbz; \bbbr^{2n}\right)\ ,
\end{equation}
where $f_{o} : = T^{-1}\int_{o}^{T} f (t) dt$. For instance,
\begin{equation}
  f (t) = \sum_{k\in \bbbn} \delta_{k} \xi\ ,
\end{equation}
where $\delta_{k}$ is the Dirac mass at $t= k$ and
$\xi \in \bbbr^{2n}$ is a
constant, fits the prescription. This means that the system
$\dot{x} = JH' (x)$ is being excited by a
series of identical shocks at interval $T$.
\end{example}
%
\begin{definition}
Let $A_{\infty} (t)$ and $B_{\infty} (t)$ be symmetric
operators in $\bbbr^{2n}$, depending continuously on
$t\in [0,T]$, such that
$A_{\infty} (t) \le B_{\infty} (t)$ for all $t$.

A Borelian function
$H: [0,T]\times \bbbr^{2n} \to \bbbr$
is called
$\left(A_{\infty} ,B_{\infty}\right)$-{\it subquadratic at infinity}
if there exists a function $N(t,x)$ such that:
\begin{equation}
  H (t,x) = \frac{1}{2} \left(A_{\infty} (t) x,x\right) + N(t,x)
\end{equation}
\begin{equation}
  \forall t\ ,\ \ \ N(t,x)\ \ \ \ \
  {\rm is\ convex\ with\  respect\  to}\ \ x
\end{equation}
\begin{equation}
  N(t,x) \ge n\left(\left\|x\right\|\right)\ \ \ \ \
  {\rm with}\ \ n(s)s^{-1}\to +\infty\ \ {\rm as}\ \ s\to +\infty
\end{equation}
\begin{equation}
  \exists c\in \bbbr\ :\ \ \ H (t,x) \le
  \frac{1}{2} \left(B_{\infty} (t) x,x\right) + c\ \ \ \forall x\ .
\end{equation}

If $A_{\infty} (t) = a_{\infty} I$ and
$B_{\infty} (t) = b_{\infty} I$, with
$a_{\infty} \le b_{\infty} \in \bbbr$,
we shall say that $H$ is
$\left(a_{\infty},b_{\infty}\right)$-subquadratic
at infinity. As an example, the function
$\left\|x\right\|^{\alpha}$, with
$1\le \alpha < 2$, is $(0,\varepsilon )$-subquadratic at infinity
for every $\varepsilon > 0$. Similarly, the Hamiltonian
\begin{equation}
H (t,x) = \frac{1}{2} k \left\|k\right\|^{2} +\left\|x\right\|^{\alpha}
\end{equation}
is $(k,k+\varepsilon )$-subquadratic for every $\varepsilon > 0$.
Note that, if $k<0$, it is not convex.
\end{definition}
%

\paragraph{Notes and Comments.}
The first results on subharmonics were
obtained by Rabinowitz in \cite{fo:kes:nic:tue}, who showed the existence of
infinitely many subharmonics both in the subquadratic and superquadratic
case, with suitable growth conditions on $H'$. Again the duality
approach enabled Clarke and Ekeland in \cite{may:ehr:stein} to treat the
same problem in the convex-subquadratic case, with growth conditions on
$H$ only.

Recently, Michalek and Tarantello (see \cite{fost:kes} and \cite{czaj:fitz})
have obtained lower bound on the number of subharmonics of period $kT$,
based on symmetry considerations and on pinching estimates, as in
Sect.~5.2 of this article.

%
% ---- Bibliography ----
%
\begin{thebibliography}{6}
%

\bibitem {smit:wat}
Smith, T.F., Waterman, M.S.: Identification of common molecular subsequences.
J. Mol. Biol. 147, 195?197 (1981). \url{doi:10.1016/0022-2836(81)90087-5}

\bibitem {may:ehr:stein}
May, P., Ehrlich, H.-C., Steinke, T.: ZIB structure prediction pipeline:
composing a complex biological workflow through web services.
In: Nagel, W.E., Walter, W.V., Lehner, W. (eds.) Euro-Par 2006.
LNCS, vol. 4128, pp. 1148?1158. Springer, Heidelberg (2006).
\url{doi:10.1007/11823285_121}

\bibitem {fost:kes}
Foster, I., Kesselman, C.: The Grid: Blueprint for a New Computing Infrastructure.
Morgan Kaufmann, San Francisco (1999)

\bibitem {czaj:fitz}
Czajkowski, K., Fitzgerald, S., Foster, I., Kesselman, C.: Grid information services
for distributed resource sharing. In: 10th IEEE International Symposium
on High Performance Distributed Computing, pp. 181?184. IEEE Press, New York (2001).
\url{doi: 10.1109/HPDC.2001.945188}

\bibitem {fo:kes:nic:tue}
Foster, I., Kesselman, C., Nick, J., Tuecke, S.: The physiology of the grid: an open grid services architecture for distributed systems integration. Technical report, Global Grid
Forum (2002)

\bibitem {onlyurl}
National Center for Biotechnology Information. \url{http://www.ncbi.nlm.nih.gov}


\end{thebibliography}
\end{document}
